%%%%%%%%%%%%%%%%%
% This is an sample CV template created using altacv.cls
% (v1.6, 21 May 2021) written by LianTze Lim (liantze@gmail.com). Now compiles with pdfLaTeX, XeLaTeX and LuaLaTeX.
%
%% It may be distributed and/or modified under the
%% conditions of the LaTeX Project Public License, either version 1.3
%% of this license or (at your option) any later version.
%% The latest version of this license is in
%%    http://www.latex-project.org/lppl.txt
%% and version 1.3 or later is part of all distributions of LaTeX
%% version 2003/12/01 or later.
%%%%%%%%%%%%%%%%

%% Use the "normalphoto" option if you want a normal photo instead of cropped to a circle
% \documentclass[10pt,a4paper,normalphoto]{altacv}

\documentclass[10pt,a4paper,ragged2e,withhyper]{altacv}
%% AltaCV uses the fontawesome5 and packages.
%% See http://texdoc.net/pkg/fontawesome5 for full list of symbols.

% Change the page layout if you need to
\geometry{left=0.5cm,right=0.5cm,top=0.5cm,bottom=0.5cm,columnsep=0.5cm}

% The paracol package lets you typeset columns of text in parallel
\usepackage{paracol}

% Change the font if you want to, depending on whether
% you're using pdflatex or xelatex/lualatex
\ifxetexorluatex
  % If using xelatex or lualatex:
  \setmainfont{Carlito}
  \setsansfont{Lato}
  \renewcommand{\familydefault}{\sfdefault}
\else
  % If using pdflatex:
  \usepackage[rm]{roboto}
  \usepackage[defaultsans]{lato}
  % \usepackage{sourcesanspro}
  \renewcommand{\familydefault}{\sfdefault}
\fi


% Change the colours if you want to
\definecolor{VividPurple}{HTML}{140E77}
\definecolor{SlateGrey}{HTML}{2E2E2E}
\definecolor{LightGrey}{HTML}{666666}
\colorlet{heading}{VividPurple}
\colorlet{accent}{VividPurple}
\colorlet{emphasis}{SlateGrey}
\colorlet{body}{LightGrey}

% Change some fonts, if necessary
\renewcommand{\namefont}{\Huge\rmfamily\bfseries}
\renewcommand{\personalinfofont}{\footnotesize}
\renewcommand{\cvsectionfont}{\LARGE\rmfamily\bfseries}
\renewcommand{\cvsubsectionfont}{\large\bfseries}


% Change the bullets for itemize and rating marker
% for \cvskill if you want to
\renewcommand{\itemmarker}{{\small\textbullet}}
\renewcommand{\ratingmarker}{\faCircle}

%% Use (and optionally edit if necessary) this .cfg if you
%% want an originally numerical reference style like IEEE
%% for your publication list
% \input{pubs-num.cfg}

\begin{document}
\name{Haotian Zhou}
\tagline{Software Engineer @ Amazon | MSAII Alumni @ CMU}
%% You can add multiple photos on the left or right
% \photoR{2.8cm}{photo}
% \photoL{2.5cm}{Yacht_High,Suitcase_High}

\personalinfo{%
  % Not all of these are required!
  \email{haotian2@alumni.cmu.edu}
  \phone{412-708-3839}
  \location{Seattle, USA}
  \linkedin{https://www.linkedin.com/in/haotian-zhou/}
  %% You can add your own arbitrary detail with
  %% \printinfo{symbol}{detail}[optional hyperlink prefix]
  % \printinfo{\faPaw}{Hey ho!}[https://example.com/]
  %% Or you can declare your own field with
  %% \NewInfoFiled{fieldname}{symbol}[optional hyperlink prefix] and use it:
  % \NewInfoField{gitlab}{\faGitlab}[https://gitlab.com/]
  % \gitlab{your_id}
  %%
  %% For services and platforms like Mastodon where there isn't a
  %% straightforward relation between the user ID/nickname and the hyperlink,
  %% you can use \printinfo directly e.g.
  % \printinfo{\faMastodon}{@username@instace}[https://instance.url/@username]
  %% But if you absolutely want to create new dedicated info fields for
  %% such platforms, then use \NewInfoField* with a star:
  % \NewInfoField*{mastodon}{\faMastodon}
  %% then you can use \mastodon, with TWO arguments where the 2nd argument is
  %% the full hyperlink.
  % \mastodon{@username@instance}{https://instance.url/@username}
}

\makecvheader
%% Depending on your tastes, you may want to make fonts of itemize environments slightly smaller
% \AtBeginEnvironment{itemize}{\small}

%% Set the left/right column width ratio to 6:4.
\columnratio{0.68}

% Start a 2-column paracol. Both the left and right columns will automatically
% break across pages if things get too long.
\begin{paracol}{2}
  \cvsection{Experience \& Projects}

  \cvcorp{Amazon}{SDE Fulltime}{Jul 2020 - Present}{Seattle}

  \cvproject{Seller LeveL CIV\footnote[1]{CIV : Consumer Instock Value, an economic concept that evaluates long-term impact of a retail offer instock status change} for Amazon.in} {Provide CIV estimates for more than 120,000 sellers in India}
  \begin{itemize}
    \item Build and automate backend data pipeline which includes . automated the pipeline based on AWS Step Function in Python
    \item Work together with economists to develop machine learning model for Seller Level CIV.
    \item Design redshift table; implement ETL jobs in hql and scala to load data via EMR
    \item Refactor RESTful Service API to support seller-level granularity. use cache to increase ...
  \end{itemize}

  \cvproject{CIV Data Pipeline Infrastructure Optimization} {Greatly improved the performance, reliability and scalability of internal systems}
  \begin{itemize}
    \item Migrate DynamoDB related operations from single host to EMR cluster. Distribute DDB operations to all executors by using Spark, 7 times faster.
    \item Implement AWS Lambda Function support in current infrastructure, migrate backend system from hosts to serverless application.
  \end{itemize}

  \cvproject{Contribution Profit Boost Process Automation} {Automate the CP Boost process cross three teams (CIV, Finance, data provider)}
  \begin{itemize}
    \item Coordinate \& implement this automated pipeline across multiple teams(...)
    \item use S3, StepFunction, Pandas, save 70\% of manual works during this process
    \item Reduced manual operation by 80\%. Shortened the entire process from one month to 5 days
  \end{itemize}

  \divider

  \cvcorp{Amazon}{SDE Intern}{May 2019 - Aug 2019}{Seattle}

  \cvproject{CIV Backend 2.0 Prototype} {Designed and implemented extendable Data pipeline backend built fully on Native AWS}
  \begin{itemize}
    \item Built data pipeline backend with centralized workflow control based on AWS Step Function
    \item Developed new features by integrating multiple AWS Services : Lambda Function, CloudWatch, SNS, EMR
  \end{itemize}

  \divider

  \cvcorp{Carnegie Mellon University }{Capstone Project }{Jan 2020 - May 2020}{Pittsburgh}

  \cvproject{Chatbot to Chatbot QA system  }{Automated chatbot that can make outbound call to IVR\footnote[2]{IVR: Interactive Voice Response System} system, answer IVR questions \& transfer call back to human}
  \begin{itemize}
    \item Integrated with Twillio APIs to make outbound / inbound call and get audio streams
    \item Integrate with Google Cloud Speech-to-Text API to transcribe audit stream in real-time
    \item Applied NLP technologies to detect questions, search \& build answers according to information stored in database
    \item Sponsor Scholarship Winner
  \end{itemize}

  %% Switch to the right column. This will now automatically move to the second
  %% page if the content is too long.
  \switchcolumn

  \cvsection{Career Objective}
  % SDE :
  
  % MLE :
  % Data engineer :
  % Distributed Sys :


  \cvsection{Area Of Interest}

  \cvtag{Software Development}
  \cvtag{NLP}
  \cvtag{Cloud Computing}
  \cvtag{Data Engineering}
  \cvtag{Machine Learning}
  \cvtag{Deep Learning}

  \cvsection{Technical SKills}

  \cvskill{Python}{4}
  \cvskill{Java}{4}
  \cvskill{C}{4}
  \cvskill{Scala}{3}
  \cvskill{JavaScript}{3}
  \cvskill{Apache Ant}{3}
  \cvskill{SQL \& HQL}{4}
  \par
  \bigskip\normalsize

  \cvskill{AWS}{5}
  \cvskill{Git}{4}
  \cvskill{Spark \& Hadoop}{4}
  \cvskill{DeepLearning}{4}
  \cvskill{NLP}{4}
  \cvskill{Distributed System}{3}

  % \cvtag{Motivator \& Leader}

  \cvsection{Personal SKills}
  \cvtag{Quick Learner}
  \cvtag{Teamworking}
  \cvtag{Work well under pressure}

  \cvsection{Education}

  \cvedu{M.S. Artificial Intelligence}{Carnegie Mellon University}{2018-2020}{GPA: 3.95}
  \cvedu{B.E. Computer Science}{Harbin Institute of Technology}{2014-2018}{GPA: 3.75}



\end{paracol}


\end{document}
